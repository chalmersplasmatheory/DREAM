\documentclass{notes}

\usepackage{overpic}
\usepackage{siunitx}

\title{Magnetic moment conservation for electrons in a time-varying magnetic field}
\author{Mathias Hoppe}
\date{2022-10-30}

\newcommand{\Bmin}{B_{\mathrm{min}}}
\newcommand{\ee}{\mathrm{e}}
\newcommand{\Vp}{\mathcal{V}'}

\newcommand{\ppar}{p_\parallel}
\newcommand{\pperp}{p_\perp}

\begin{document}
	\maketitle

	Because of the conservation of magnetic moment, when the magnetic field is
	varied, the electrons' pitch angles will change. The magnetic moment for a
	relativistic electron is
	\begin{equation}
		\mu = \frac{mc^2\pperp^2}{2B},
	\end{equation}
	where $m$ denotes the electron rest mass and $\pperp$ is normalized to $mc$.
	Consequently, for $\mu$ to be conserved, the perpendicular momentum $\pperp$
	must change when $B$ changes according to
	\begin{equation}
		\pperp = \sqrt{\frac{2\mu B}{mc^2}}.
	\end{equation}
	Alternatively, since the total momentum $p$ is conserved, we can express
	this in terms of the pitch:
	\begin{equation}
		\xi = \sqrt{1-\frac{2\mu B}{mc^2p^2}}.
	\end{equation}
	The time-rate-of-change of $\pperp$ and $\xi$ are therefore
	\begin{equation}
		\begin{aligned}
			\frac{\partial\pperp}{\partial t} &= \frac{\mu}{c\sqrt{2m\mu B}}
			\frac{\partial B}{\partial t} =
			\frac{\mu}{mc^2\pperp}\frac{\partial B}{\partial t},\\
			%
			p\frac{\partial\left|\xi\right|}{\partial t} &= -\frac{\mu}{c\sqrt{(mcp)^2-2m\mu B}}
			\frac{\partial B}{\partial t} =
			-\frac{\mu}{mc^2\left|\ppar\right|}\frac{\partial B}{\partial t}.
		\end{aligned}
	\end{equation}
	These are also the effective forces acting on all particles in the $\pperp$
	and $\ppar$ directions (normalized to $mc$),
	\begin{equation}
		\begin{aligned}
			F_\perp &= \frac{\partial\pperp}{\partial t} =
			\frac{\mu}{c\pperp}\frac{\partial B}{\partial t},\\
			%
			F_\parallel &= \frac{\partial\left|\ppar\right|}{\partial t} =
			-\frac{\mu}{c\left|\ppar\right|}\frac{\partial B}{\partial t}.
		\end{aligned}
	\end{equation}
	The parallel force given here explicitly affects the magnitude of the
	parallel momentum. To correctly account for the sign of the force we can
	multiply it by $\mathrm{sgn}\,\ppar$, which cancels the modulus in the
	denominator.

	To consider what would happen to a population of electrons, we need to
	evaluate the advection coefficients in $(p,\xi)$ corresponding to the force
	above. The advection coefficient in the direction of a coordinate $z^m$ is
	generally given by
	\begin{equation}
		A^m = \nabla z^m\cdot\bb{F}.
	\end{equation}
	For $(p,\xi)$ coordinates, the relevant contravariant basis vectors are
	\begin{equation}
		\nabla p = \hat{p},\quad \nabla\xi = -\frac{\sqrt{1-\xi^2}}{p}\hat{\xi},
	\end{equation}
	where the unit vectors $\hat{p}$ and $\hat{\xi}$, as well as the unit
	vectors $\hat{p}_\parallel$ and $\hat{p}_\perp$ which we have implicitly
	expressed the force in so far, are related via
	\begin{equation}
		\begin{aligned}
			\hat{p}_\parallel &= \hat{z}, \quad &\hat{p} = \sqrt{1-\xi^2}\left(\hat{x}\cos\varphi + \hat{y}\sin\varphi\right),\\
			\hat{p}_\perp &= \hat{x}\cos\varphi + \hat{y}\sin\varphi, \quad &\hat{\xi} = \xi\left(\hat{x}\cos\varphi + \hat{y}\sin\varphi\right) - \sqrt{1-\xi^2}\hat{z},
		\end{aligned}
	\end{equation}
	with scalar products
	\begin{equation}
		\begin{aligned}
			\hat{p}_\parallel\cdot\hat{p} &= \xi, \quad
			&\hat{p}_\perp\cdot\hat{p} &= \sqrt{1-\xi^2},\\
			\hat{p}_\parallel\cdot\hat{\xi} &= -\sqrt{1-\xi^2},\quad
			&\hat{p}_\perp\cdot\hat{\xi} &= \xi.
		\end{aligned}
	\end{equation}
	The advection coefficients in $(p,\xi)$ are therefore
	\begin{equation}\label{eq:Fmu}
		\begin{aligned}
			A^p &= F_\parallel\left(\hat{p}_\parallel\cdot\hat{p}\right) + F_\perp\left(\hat{p}_\perp\cdot\hat{p}\right) =
			\frac{\mu}{cp}\frac{\partial B}{\partial t}\left(-\frac{\xi}{\xi} + \frac{\sqrt{1-\xi^2}}{\sqrt{1-\xi^2}}\right) = 0,\\
			A^\xi &= -\frac{\sqrt{1-\xi^2}}{p}\left[F_\parallel\left(\hat{p}_\parallel\cdot\hat{\xi}\right) + F_\perp\left(\hat{p}_\perp\cdot\hat{\xi}\right)\right] =
			-\frac{\sqrt{1-\xi^2}}{p}\frac{\mu}{cp}\frac{\partial B}{\partial t}\left(\frac{\pperp}{\ppar} + \frac{\ppar}{\pperp}\right) =\\
			&= -\frac{1-\xi^2}{\xi}\frac{mc}{2B}\frac{\partial B}{\partial t}.
		\end{aligned}
	\end{equation}
	This shows that, since $A^p=0$, the force conserves total energy and only
	transfers momentum between the parallel and perpendicular momentum
	components. When the magnetic field is decreasing, the pitch must increase,
	leading to smaller pitch angles for the electrons, and vice versa.

	\section*{Numerical estimates}
	We can estimate the relative importance of the magnetic compression force by
	comparing it to the main competing advective process, namely the electric
	field acceleration. The advection coefficient in the $\xi$ direction for the
	electric field is
	\begin{equation}
		A^\xi_E = eE\frac{\sqrt{1-\xi^2}}{p}.
	\end{equation}
	The ration between the electric and magnetic compression forces is
	therefore
	\begin{equation}
		\left|\frac{A^\xi_\mu}{A^\xi_E}\right| =
			\frac{mc}{e}
			\left|\frac{p\sqrt{1-\xi^2}}{\xi}\frac{\partial B/\partial t}{2EB}\right|
			\approx
			\frac{p\sqrt{1-\xi^2}}{587\left|\xi\right|}\left|\frac{\partial B/\partial t}{2EB}\right|
	\end{equation}
	For magnetic compression to be negligible we must have
	$A^\xi_E/A^\xi_\mu\ll 1$. In TCV, typical values for $E$, $B$ and
	$\partial B/\partial t$ are
	\begin{equation*}
		\begin{aligned}
			E\approx\SI{0.1}{V/m},\\
			B\approx\SI{1.4}{T},\\
			\frac{\partial B}{\partial t}\approx\SI{1}{T/s}.
		\end{aligned}
	\end{equation*}
	Furthermore, the runaways contributing to synchrotron emission tend to have
	$p\approx 50mc$ and $\xi\approx 0.85$, giving
	\begin{equation}
		\left|\frac{A^\xi_\mu}{A^\xi_E}\right| \approx
			0.18,
	\end{equation}
	suggesting that the magnetic compression force on the pitch angles should be
	small compared to the effect of the electric field.

	\section*{Bounce average}
	To implement the magnetic compression force~\eqref{eq:Fmu} into \DREAM\ we
	first need to bounce-average it. The bounce average operator is
	\begin{equation}
		\left\{\cdots\right\} = \frac{4\pi^2}{\Vp}
			\oint\dd\theta\,p^2\frac{B}{\Bmin}\frac{\xi_0}{\xi}\mathcal{J},
	\end{equation}
	where $\Vp$ is the phase-space jacobian and $\mathcal{J}$ is the
	configuration space jacobian. Applying this operator to the compression
	force yields
	\begin{equation}
		\begin{aligned}
			\left\{ A^\xi \right\} =
				\frac{mc}{2}\left\{\frac{1-\xi^2}{\xi}\frac{1}{B}\frac{\partial B}{\partial t}\right\}
		\end{aligned}
	\end{equation}
	Since, at the time of writing this, \DREAM\ does not support time-evolving
	magnetic fields, we should replace $B^{-1}(\partial B/\partial t)$ with a
	parameter which can be straightforwardly specified by the user. Taking
	$B\approx B_0R_0/R$, where $B_0$ is the magnetic field strength at $R=R_0$,
	the advection coefficient can be written
	\begin{equation}
		\left\{A^\xi\right\}\approx\frac{mc}{2B_0}\frac{\partial B_0}{\partial t}
		\left\{\frac{1-\xi^2}{\xi}\right\},
	\end{equation}
	where the remaining bounce-average has to be evaluated numerically.

	\section*{Effect in TCV}
	The effect of the magnetic compression force on the average pitch angle of
	runaway electrons is illustrated in figure~\ref{fig:TCV}.
	\begin{figure}
		\centering
		\begin{overpic}[width=\textwidth]{figs/TCV_avtheta_magcompr.pdf}
			\put(44,72){(a)}
			\put(92,72){(b)}
			\put(44,29){(c)}
			\put(92,29){(d)}
		\end{overpic}
		\caption{
			Effect on the average pitch angle of the magnetic compression force
			when the toroidal magnetic field is decreased at three different
			rates:
			$\partial B/\partial t=-\SI{1}{T/s}$,
			$\partial B/\partial t=-\SI{3}{T/s}$, and
			$\partial B/\partial t=-\SI{6}{T/s}$. Panel (a) shows the absolute
			change in the average pitch angle $\langle\theta\rangle$, as a
			function of momentum, while (c) shows the change in
			$\langle\theta\rangle$ relative to the case without a magnetic
			field ramp. In panels (b) and (d), the absolute and relative change
			in average pitch angle at $p=37.5mc$ is shown as a function of time,
			illustrating that a steady-state in the pitch angle shift is
			quickly reached. In panels (b) and (d), the vertical dotted line
			indicates the approximate time at which the first runaway electrons
			reach the considered momentum.
		}
		\label{fig:TCV}
	\end{figure}

	%\section*{Analytical estimate}
	%The effect of the magnetic compression in a runaway plateau can be estimated
	%using the same approach as Aleynikov and Breizman used for calculating the
	%pitch angle distribution. When pitch angle scattering and electric field
	%acceleration (and magnetic compression) approximately balance each other,
	%the kinetic equation can be written
	%\begin{equation}
	%	\frac{\partial f}{\partial t} = \frac{\partial}{\partial\xi}\left[
	%		\left(1-\xi^2\right)\left(
	%			-\frac{E}{p}f -
	%			\frac{\sqrt{1-\xi^2}}{\xi}\frac{\beta}{2} f +
	%			\frac{\nu_D(p)}{2}\frac{\partial f}{\partial\xi}
	%		\right)
	%	\right],
	%\end{equation}
	%where we introduced $\beta\equiv B_0^{-1}(\partial B_0/\partial t)$ as it
	%will prove convenient in the later analysis. Taking the steady-state limit,
	%we first obtain
	%\begin{equation}
	%	\begin{gathered}
	%		-\frac{E}{p}f -
	%		\frac{\sqrt{1-\xi^2}}{\xi}\frac{\beta}{2} f +
	%		\frac{\nu_D(p)}{2}\frac{\partial f}{\partial\xi}
	%		= 0\\
	%		%
	%		\Longleftrightarrow\\
	%		%
	%		\left(\frac{E}{p} + \frac{\sqrt{1-\xi^2}}{\xi}\frac{\beta}{2}\right)f =
	%		\frac{\nu_D}{2}\frac{\partial f}{\partial\xi},
	%	\end{gathered}
	%\end{equation}
	%Substituting $a=2E/p\nu_D$ and $b=\beta/(2\nu_D)$,
	%we can solve this equation through
	%\begin{equation}
	%	\begin{gathered}
	%		\left(a+2b\frac{\sqrt{1-\xi^2}}{\xi}\right)f = \frac{\partial f}{\partial\xi}\\
	%		%
	%		\implies\\
	%		%
	%		\int_{f(-1)}^{f(\xi)} \frac{\dd f}{f}
	%		=
	%		\int_{-1}^\xi\left(a + 2b\frac{\sqrt{1-x^2}}{x}\right)\,\dd x\\
	%		%
	%		\implies\\
	%		%
	%		\ln\frac{f(\xi)}{f(-1)} =
	%		\left[a(\xi+1) + 2b\sqrt{1-\xi^2} + b\ln\frac{1-\sqrt{1-\xi^2}}{1+\sqrt{1-\xi^2}}\right]\\
	%		%
	%		\implies\\
	%		%
	%		%f(\xi) = f(-1)\left|\xi\right|^{2b}\exp\left[ a(\xi+1) + b\left(1-\xi^2\right)\right] =\\
	%		f(\xi) = f(-1)\left(\frac{1-\sqrt{1-\xi^2}}{1+\sqrt{1-\xi^2}}\right)^b
	%		\exp\left[a\left(\xi+1\right) + 2b\sqrt{1-\xi^2}\right]
	%	\end{gathered}
	%\end{equation}
	%Gathering terms independent of $\xi$ in the exponent into the pre-factor,
	%the distribution can finally be written
	%\begin{equation}\label{eq:fana}
	%	f(\xi) = f_0\left(\frac{1-\sqrt{1-\xi^2}}{1+\sqrt{1-\xi^2}}\right)^b
	%	\ee^{a\xi+2b\sqrt{1-\xi^2}}.
	%\end{equation}
	%Next, we will estimate how the emitted synchrotron power depends on $\beta$.
	%The emitted synchrotron power is proportional to $p_\perp^2$, so the total
	%power emitted by~\eqref{eq:fana} at a given value of momentum is
	%proportional to
	%\begin{equation}
	%	P = \int_{-1}^1\left(1-\xi^2\right)f\left(\xi\right)\,\dd\xi.
	%\end{equation}

\end{document}
