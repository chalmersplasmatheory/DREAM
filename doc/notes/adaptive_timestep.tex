\documentclass{notes}

\title{Adaptive time steps in DREAM}
\author{Mathias Hoppe}

\newcommand{\Deltats}{\Delta t^\star}
\newcommand{\epst}{\epsilon_{\Delta t}}
\newcommand{\epstol}{\epsilon_{\rm tol}}
\newcommand{\Eps}{\mathcal{E}}

\begin{document}
    \maketitle

    \noindent
    The solution to the \DREAM\ equations can evolve on a wide range of time
    scales, even during the same simulation. For example, the thermal quench
    and current quench are usually separated by one or more orders of magnitude,
    as are the current quench and runaway plateau. To ensure that the simulation
    is always well-resolved, without requiring the user to manually split the
    simulation into separate and carefully adjust parameters, \DREAM\ provides
    functionality for manually adjusting the time step during the simulation.
    This document describes the theoretical basis for this adaptive timestepper,
    as well as the details of how it is implemented in \DREAM.
    
    \section{Theory}
    The purpose of the adaptive time stepper is to balance two objectives: first
    and foremost, we want the solution to be correct and accurate, and so the
    adaptive time stepper must ensure that the truncation error
    $\epst$ is always sufficiently small, i.e. below some set
    tolerance $\epstol$. We could always ensure this by keeping the time step
    $\Delta t$ fixed and have $\Delta t\to 0$, but this would also cause the
    wall time of the simulation to approach infinity. Rather, we would prefer
    the simulation to finish as quickly as possible. Therefore, the second
    objective that the adaptive time stepper must achieve is to make $\Delta t$
    as large as possible, since the wall time of the simulation can usually be
    assumed to be proportional to $\Delta t$.

    The way to balance these two objectives is to, first, always ensure that
    $\epst\leq\epstol$. When $\epst<\epstol$, this indicates that the time step
    $\Delta t$ is a bit too small and we could increase it without violating the
    accuracy condition.
    
    In order to devise an adaptive time stepper we need to determine two
    quantities, namely the truncation error $\epst$ in any given time step,
    as well as the optimal time step $\Deltats(\epst)$ to take. The bible,
    commonly also known as \emph{Numerical recipes}, provides the basic building
    blocks for both of these.

    We will determine the local error $\epst$ by taking a total of three steps.
    First, we take a full step of length $\Delta t$ in order to obtain the
    solution $y_1$. Next, we revert the solver and take two separate steps, each
    of size $\Delta t/2$, and obtain the solution $y_2$. The difference between
    the two solutions can be taken as the truncation error:
    \begin{equation}
        \epst\equiv \left| y_2 - y_1 \right|.
    \end{equation}
    For checking convergence, we evaluate this truncation error for each unknown
    quantity separately. In order to adjust the time step, however, we select
    the largest error:
    \begin{equation}
        \Eps = \max_i\frac{\epst^{(i)}}{\epstol^{(i)}},
    \end{equation}
    where $i$ identifies the unknown quantity. If $\Eps > 1$, the error is too
    large, and vice versa. If the step taken to achieve a too large error
    $\Eps_1$ was $\Delta t_1$, and we strive for an error $\Eps^\star$, then the time step
    that would have given the correct error (in Euler backward) is
    \begin{equation}
        \Deltats = \Delta t_1\left|\frac{\Eps^\star}{\Eps_1}\right|.
    \end{equation}
    In \DREAM, we take $\Eps^\star = 1$ (note that this does not mean that enforce
    \emph{no error}, but rather that we require the error to be within the
    previously enforced tolerances). Since the error estimate is not exact, we
    also introduce a safety factor $S\approx0.95$ and obtain the updated time
    step
    \begin{equation}\label{eq:timestep}
        \Deltats_{n+1} = S\frac{\Delta t_n}{\Eps_n}.
    \end{equation}

    \paragraph{Improved stability}
    The stability of the stepper can be somewhat improved by noting
    that~\eqref{eq:timestep} can be considered as an
    \emph{integrating controller} in standard control theory. The $1/\Eps$
    indicates that the control variable is obtained by ``integrating'' the
    error. It is well known in control theory that in order to improve the
    stability of this simple PI controller, one can incorporate the error in
    the previous step as well:
    \begin{equation}
        \Deltats_{n+1} = S\Delta t_n \frac{\Eps_{n-1}^\beta}{\Eps_n^{\alpha}}.
    \end{equation}
    The exponents $\alpha$ are typically chosen as
    \begin{equation}
        \beta\approx 0.4/k,\quad \alpha\approx 1/k-0.75\beta = 0.7/k,
    \end{equation}
    where $k$ denotes the order of the method ($k=1$ for backward Euler). Hence,
    in \DREAM, we use $\alpha = 0.4$ and $\alpha = 0.7$.

\end{document}
