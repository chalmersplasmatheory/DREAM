\documentclass{notes}
\usepackage{physics}

\usepackage[
    backend=biber,
    sorting=none,
    style=numeric-comp,
    citestyle=numeric-comp
]{biblatex}
\addbibresource{ref.bib}

\title{Bootstrap current}
\author{Peter Halldestam}

\newcommand*\diff{\mathop{}\!\mathrm{d}}

\newcommand{\pe}{p_{\rm e}}
\newcommand{\pii}{p_{\rm i}}
\newcommand{\te}{T_{\rm e}}
\renewcommand{\ne}{n_{\rm e}}
\newcommand{\tii}{T_{\rm i}}
\newcommand{\nii}{n_{\rm i}}
\newcommand{\wii}{w_{\rm i}}
\newcommand{\isum}{\sum_{\rm ions}}

\newcommand{\johm}{j_\Omega}

\newcommand{\nui}{\nu_{\rm i}^*}
\newcommand{\nue}{\nu_{\rm e}^*}
\newcommand{\Zeff}{Z_{\rm eff}}
\newcommand{\ft}{f_{\rm t}}
\renewcommand{\L}[1]{\mathcal{L}_{#1}}

\newcommand{\cond}{\sigma_\parallel}
\newcommand{\jpara}{j_\parallel}
\newcommand{\Epara}{E_\parallel}
\newcommand{\jbs}{j_{\rm bs}}
\newcommand{\fsa}[1]{\left\langle #1 \right\rangle}

\begin{document}
    \maketitle

    \noindent
    The set of analytical formulae describing the non-inductive bootstrap current $\jbs$, presented together with a neoclassical correction to the Spitzer conductiviy in the Redl-Sauter paper \cite{Redl2021}, is implemented in DREAM for studying SPI in disrupting H-mode plasmas.
    Both bootstrap current is given in the following form
    \begin{align}
        \label{eq:redl-sauter}
        \fsa{\jbs B} =
            -G(\psi)\left[\L{31}p\pdv{\ln{n}}{\psi} + (\L{31}+\L{32})\pe\pdv{\ln{\te}}{\psi} + \L{31}(1+\alpha)\isum\pii\pdv{\ln{\tii}}{\psi}\right],
    \end{align}
    where $\fsa{\vdot}$ denotes the flux surface average as function of the poloidal magnetix flux $\psi$, $B$ is the magnetic field strength and $G(\psi)=RB_\phi$.
    The dynamic quantities that can be evolved in \DREAM\ that appears are the electron and ion density $\ne$ and $\nii$ (summed over all charge states per ion species), their temperatures $\te$ and $\tii$, as well as the total density and pressure, given by
    \begin{align}
        \label{eq:total-density-pressure}
        n=\ne+\isum\nii,\qquad p=\pe+\isum\pii=\ne\te+\frac{2}{3}\isum\wii.
    \end{align}
    Note that since the ion temperatures are evolved via their thermal energy density in \DREAM\, we will make use the relation of the ideal gas $\wii=3\pii/2=3\nii\tii/2$.
    We will therefore set up discretisation schemes for differentiating the four quantities $\ne$, $\te$, $\nii$ and $\wii$.
    Furthermore, the coefficients $\L{31}$, $\L{32}$ and $\alpha$ in Eq.~\eqref{eq:redl-sauter} are defined in Redl \textit{et al.} and are functions of the fraction of trapped particles $\ft(\psi)$, the effective charge number $\Zeff(\psi)$ and the electron collisionality $\nue(\psi)$ (for $\alpha$ it is rather the ion collisionality $\nui(\psi)$ of the main ion species).
    % These plasma parameters, save $\ft$, are all dependent on the four evolving quantities, which must be taken into account when

    Similar to the other current densities in \DREAM\, we assume the bootstrap current to scale with the magnetic field strength $\jbs\sim B(\theta)$ such that $\jbs/B$ is constant on flux surfaces.
    Therefore, we have
    \begin{align}
        \fsa{\jbs B} = \fsa{\frac{\jbs}{B}B^2} = \frac{\jbs}{B}\fsa{B^2}.
    \end{align}
    With the coordinate transform $\partial/\partial{\psi}=1/\psi'\partial/\partial{r}$, where $\psi'$ is the radial derivative of the (reference) poloidal flux which is held constant in time witin this framework, the formula in Eq.~\eqref{eq:redl-sauter} is turned into
    \begin{align}
        \frac{\jbs}{B}
            &=-\frac{G}{\fsa{B^2}\psi'}\left[\L{31}p\pdv{\ln{n}}{r} + (\L{31}+\L{32})\pe\pdv{\ln{\te}}{r} + \L{31}(1+\alpha)\isum\pii\pdv{\ln{\tii}}{r}\right].
    \end{align}
    Seperating into terms proportional to derivatives of the four quantities $\ne$, $\te$, $\nii$ and $\wii$ leads to
    \begin{align}
        \label{eq:implemented1}
        \frac{\jbs}{B} = -\frac{G}{\fsa{B^2}\psi'}\bigg\{
            \L{31}\frac{p}{n}\pdv{\ne}{r}
            &+\L{31}\isum\left[\frac{p}{n}-(1+\alpha)\frac{2\wii}{3\nii}\right]\pdv{\nii}{r}\notag\\
            &+(\L{31}+\L{32})\ne\pdv{\te}{r}
            +\L{31}(1+\alpha)\frac{2}{3}\isum\pdv{\wii}{r}\bigg\}.
    \end{align}
    In simulations where the ion temperature is not accounted for, we assume $\tii=\te$ and Eq.~\eqref{eq:implemented1} is expressed as
    \begin{align}
        \label{eq:implemented2}
        \frac{\jbs}{B} = -\frac{G}{\fsa{B^2}\psi'}\bigg\{
            &\L{31}\frac{p}{n}\pdv{\ne}{r}
            +\L{31}\frac{p}{n}\isum\pdv{\nii}{r}\notag\\
            &+\left[(\L{31}+\L{32})\ne+\L{31}(1+\alpha)\isum\nii\right]\pdv{\te}{r}\bigg\}.
    \end{align}
    Both these equations in Eqs.~\eqref{eq:implemented1} and \eqref{eq:implemented2} have their radial derivatives discretised in the same way.
    For any general quantity $X$, the central difference approximation is used as
    \begin{align}
        \label{eq:central-difference}
        \pdv{X}{r}\bigg|_{r_{k}}
            \approx\frac{X_{k+1}-X_{k-1}}{r_{k+1}-r_{k-1}}
            =\frac{X_{k+1}-X_{k-1}}{\Delta r_{k+1/2}+\Delta r_{k-1/2}}
    \end{align}
    for any radial grid point indexed $k=1,\dots N_r$ (note that $r_1\neq0$ as this is the \textit{cell grid} and not the \textit{flux grid}).
    At the inner boundary ($k=1$) we use that $X$ is a flux surface function with the symmetry property $X(r)=X(-r)$, and for the outer boundary we switch from the central difference to a backward Euler scheme.
    The boundary conditions is thus set to
    \begin{align}
        \pdv{X}{r}\bigg|_{r_1}
            \approx\frac{X_{2}-X_1}{2\Delta r_{3/2}},\qquad
        \pdv{X}{r}\bigg|_{r_{N_r}}
            \approx\frac{X_{N_r}-X_{N_r-1}}{\Delta r_{N_r-1}}
    \end{align}

    \paragraph{Note on the safety factor:}
    The two collisionalities, $\nue$ and $\nui$, are both proportional to the safety factor $q$ in their definitions Eq.~(18b) and (18c) in the original Sauter paper \cite{Sauter1999}.
    In \DREAM, one may calculate safety factor using the already implemented method \texttt{RadialGrid::SafetyFactorNormalized}, which uses a formula (see doc/notes/theory.tex under the ``Safety factor'' section) that explicitly relates $q$ to be inversily proportional to the total current $I(r)$ enclosed by a flux surface labeled $r$.
    However, when instead using the approximation (found in page 167 of Wesson's ``Tokamaks 2nd ed.''), namely
    \begin{align}
        \label{eq:safety-factor-wesson}
        q(r)\approx\varepsilon\frac{B}{B_\theta}=\varepsilon\sqrt{1+\frac{4\pi^2G^2}{\psi'^2}},
    \end{align}
    where $\varepsilon= r/R$ is the inverse aspect ratio, a much better agreement with the implemented formula is observed when cross-checking with reconstructions of the bootstrap current in AUG discharges (AUGD/IDF which uses the Redl-Sauter model).
    \textcolor{red}{Whether we want to use one or the other formulas for the safety factor is to be determined, but for the moment, we use Eq.~\eqref{eq:safety-factor-wesson}.}

    \addcontentsline{toc}{section}{References}
    \printbibliography

\end{document}
