\documentclass[11pt,a4paper,english
]{article}
\pdfoutput=1

\usepackage{babel}

\usepackage{amsmath}
\usepackage{amssymb}
\usepackage{amsthm}
\usepackage{textcomp}
\usepackage{xcolor}
\usepackage{graphicx}
\usepackage[hidelinks, colorlinks,urlcolor=blue]{hyperref}

\usepackage{physics} 
\usepackage{datetime2} % To make the date ISO formatted

%% Partial derivative 'd':
\newcommand{\pd}{\ensuremath{\partial}}
%% 'e' and 'i' should be straight:
\newcommand{\ee}{\mathrm{e}}
\newcommand{\ii}{\mathrm{i}}

\newcommand{\nRE}{\ensuremath{n_{\rm RE}}}
\newcommand{\nuD}{\ensuremath{\nu_{\rm D}}}
\newcommand{\nus}{\ensuremath{\nu_{\rm s}}}


%% Change the margin in the documents
\usepackage[
            top    = 3cm,              %% top margin
            bottom = 3cm,              %% bottom margin
            left   = 2.5cm, right  = 2.5cm %% left and right margins
]{geometry}


%%%%%%%%%%%%%%%%%%%%%%%%%%%%%%%%%%%%%%%%%%%%%%%%%%%%%%%%%%%%%%%%%%%%%%
\begin{document}%% v v v v v v v v v v v v v v v v v v v v v v v v v v
%%%%%%%%%%%%%%%%%%%%%%%%%%%%%%%%%%%%%%%%%%%%%%%%%%%%%%%%%%%%%%%%%%%%%%


%%%%%%%%%%%%%%%%%%%% vvv Internal title page vvv %%%%%%%%%%%%%%%%%%%%%
\title{Implementation of the transport equations of 
 Pontus' paper into DREAM}
\author{Andr\'eas Sundstr\"om}
\date{2021-06-10}

\maketitle
%%%%%%%%%%%%%%%%%%%% ^^^ Internal title page ^^^ %%%%%%%%%%%%%%%%%%%%%



\section{Theoretical starting point}
We take our starting point in eq.~(3.5) of Pontu's paper
[\href{https://doi.org/10.1017/S0022377820001592}{\textsc{doi}:~10.1017/S0022377820001592};
arXiv:~\url{https://arxiv.org/abs/2010.07156}]:
\begin{equation}\label{eq:reduced-kinetic}
\pdv{F}{t}+\frac{1}{\tau}\pdv{p}\qty[\Big.U(p)F]
=\frac{1}{r}\pdv{r}\qty[r\ev{\mathsf{D}}_\xi\pdv{F}{r}],
\end{equation}
which is the reduced (integrated over pitch-angle) kinetic
equation. Here, $r$ is the radial component (in cylindrical geometry),
$p$ is the momentum coordinate (normalized to $m_{\ee}c$), $F$ is the
pitch-angle-integrated distribution function such that the runaway
electron density is given by
\begin{equation}
\nRE(r,t) = \int_{p_*}^{\infty}\dd{p}F(r,p;t),
\end{equation}
$U$ is the pitch-angle averaged force (the form of which is not
important for the implementation),
% \begin{equation}
% U(p) = \frac{E}{E_{\rm c}}\coth(W)
% -\tau\qty[\frac{p\nuD}{2}+p\nus+F_{\rm br}
% +\frac{p^2\gamma\nuD}{\tau_{\rm syn}E/E_{\rm c}}\qty(\coth(W)-W^{-1})]
% \end{equation}
and $\ev{D}_{\xi}$ is the pitch-angle averaged diffusion coefficient
\begin{equation}\label{eq:ev_D_xi}
\ev{\mathsf{D}}_{\xi}=\ev{\mathsf{D}}_{\xi}(p)
=\int_{-1}^{1}\dd{\xi} \mathsf{D}(p,\xi)\frac{W\ee^{W\xi}}{2\sinh(W)}.
\end{equation}
The parameter $W=W(r,p,t)$ is the width of the distribution in pitch
angle, and is given by
\begin{equation}
W(t,r,p) = \frac{2E(r,t)/E_{\rm c}}{p\nuD(r,t)\tau(r,t)},
%\equiv \frac{\Delta{p}(r,t)}{p},
\end{equation}
where $\tau$ is the relativistic collision time and
\begin{equation}
\nuD = \tau^{-1}\frac{\gamma}{p^3}(1+Z_{\rm eff})
= \tau^{-1}\frac{\sqrt{1+p^2}}{p^3}(1+Z_{\rm eff})
\end{equation}
is the deflection (pitch-angle scattering) collision
frequency. The width in pich angle can now be written as
\begin{equation}
W(t,r,p) = \frac{e}{m_{\ee}c}\frac{2E(r,t)p^2\tau}{\sqrt{1+p^2}(1+Z_{\rm eff})}.
\end{equation}
For better numerical stability, the integrand in \eqref{eq:ev_D_xi}
can be rewritten as
\begin{equation}
\frac{\ee^{W\xi}}{2\sinh(W)}
= \frac{\ee^{W(\xi-1)}}{1-\ee^{-2W}}.
\end{equation}


\subsection{Radial fluid transport equations}
If we allow for an inhomogeneous plasma, \eqref{eq:reduced-kinetic} is
modified to
\begin{equation}\label{eq:reduced-kinetic_VD}
\pdv{F}{t}+\frac{1}{\tau}\pdv{p}\qty[\Big.U(p)F]
=\frac{1}{r}\pdv{r}\qty[r\qty(-\ev{\mathsf{A}}_\xi+\ev{\mathsf{D}}_\xi\pdv{r})F],
\end{equation}
where $\ev{\mathsf{A}}_\xi$ is the pitch-angle averaged radial
advection term, ``defined equivalently to'' \eqref{eq:ev_D_xi}. Given
these conditions (see the paper), the lowest-order solution for $F$
becomes
\begin{equation}
F_0(p,r;t) = \nRE(r,t)
\frac{\gamma_{\rm r}\tau E_{\rm c}}{E-\bar{E}_{\rm c}^{\rm eff}}
\exp(-\frac{\gamma_{\rm r}\tau E_{\rm c}(p-p_{*})}{E-\bar{E}_{\rm c}^{\rm eff}})
\equiv \frac{\nRE(r,t)}{\bar{p}_r(r,t)}\exp(-\frac{p-p_{*}}{\bar{p}_r(r,t)}),
\end{equation}
where %
\begin{equation}
\bar{p}_r(r,t)\equiv\frac{(E-\bar{E}_{\rm c}^{\rm eff})/E_{\rm c}}
{\gamma_{\rm r}\tau}
=\frac{e}{m_{\ee}c}\,\frac{E-\bar{E}_{\rm c}^{\rm eff}}
{\gamma_{\rm r}}.
\end{equation}
Under the assumption that the radial transport is small, we may uae
this distribution to evaluate the lowest-order transport term.

Integrating \eqref{eq:reduced-kinetic_VD} over $p$ gives
\begin{equation}\label{eq:n_RE}
\pdv{\nRE}{t} + \frac{1}{r}\pdv{r}
\qty[r\qty(\frac{\bar\varGamma_{0}}{a}\nRE(r,t)+\tilde\varGamma_{0}\pdv{\nRE}{r})]
=\gamma_{\rm r}\nRE,
\end{equation}
where $a$ is the minor radius of the tokamak, and the advection and
diffusion coefficients, $\bar\varGamma_{0}$ and $\tilde\varGamma_{0}$,
respectively, are given by
\begin{equation}
\begin{aligned}
\frac{\bar\varGamma_{0}}{a}\nRE+\tilde\varGamma_{0}\pdv{\nRE}{r}
=&\int_{p_*}^{\infty}\qty(\ev{\mathsf{A}}_{\xi}F_0-\ev{\mathsf{D}}_{\xi}\pdv{F_0}{r})\dd{p}\\
=&
\int_{0}^{\infty}\ev{\mathsf{A'}}_{\xi}\frac{\nRE(r,t)}{\bar{p}_r}\ee^{-p'/\bar{p}_r}\dd{p'}\\
&-\int_{0}^{\infty}\ev{\mathsf{D'}}_{\xi}
\qty(\frac{\pd_r\nRE}{\bar{p}_r}-\frac{\nRE\,\pd_r\bar{p}_r}{\bar{p}_r^2}
+\frac{\nRE}{\bar{p}_r}\frac{p'\,\pd_r\bar{p}_r}{\bar{p}_r^2})
\ee^{-p'/\bar{p}_r}\dd{p'}\\
=&\frac{\nRE(r,t)}{\bar{p}_r}
\int_{0}^{\infty}\qty[\ev{\mathsf{A'}}_{\xi} +\ev{\mathsf{D'}}_{\xi}
\qty(\frac{\pd_r\bar{p}_r}{\bar{p}_r}
-\frac{p'\,\pd_r\bar{p}_r}{\bar{p}_r^2})]
\ee^{-p'/\bar{p}_r}\dd{p'}\\
&-\frac{\pd_r\nRE}{\bar{p}_r}\int_{0}^{\infty}\ev{\mathsf{D'}}_{\xi}
\ee^{-p'/\bar{p}_r}\dd{p'},
\end{aligned}
\end{equation}
where $p'=p-p_{*}$ and $\ev{V'}_\xi=\ev{V}_\xi(p'+p_*)$ (same for
$\ev{D'}_\xi$). Individually,  $\bar\varGamma_{0}$ and
$\tilde\varGamma_{0}$ are given as
\begin{align}
\frac{\bar\varGamma_{0}}{a}
=&
\frac{1}{\bar{p}_r}
\int_{0}^{\infty}\qty[\ev{\mathsf{A'}}_{\xi} +\ev{\mathsf{D'}}_{\xi}
\frac{1}{\bar{p}_r}\qty(1-\frac{p'}{\bar{p}_r})\pdv{\bar{p}_r}{r}]
\ee^{-p'/\bar{p}_r}\dd{p'},
\\
\tilde\varGamma_{0}=&
\int_{0}^{\infty}-\frac{1}{\bar{p}_r}\ev{\mathsf{D'}}_{\xi}
\ee^{-p'/\bar{p}_r}\dd{p'}.
\end{align}
We also note that with %
$\bar{p}_r(r,t)=(E-\bar{E}_{\rm c}^{\rm eff})/\gamma_{\rm r}\tau$, %
\begin{equation}
\pdv{\bar{p}_r}{r}=
\frac{1}{\gamma_{\rm r}\tau E_{\rm c}}
\pdv{(E-\bar{E}_{\rm c}^{\rm eff})}{r}
-\frac{E-\bar{E}_{\rm c}^{\rm eff}}{(\gamma_{\rm r}\tau E_{\rm c})^2}
\qty(\gamma_{\rm r}E_{\rm c}\pdv{\tau}{r}
+\tau E_{\rm c}\pdv{\gamma_{\rm r}}{r}
+\gamma_{\rm r}\tau\pdv{E_{\rm c}}{r}).
\end{equation}
We could also express $\bar{\varGamma}_0$ in terms of
$\bar{p}_{r}^{-1}$, which means rewiting the derivative as 
\begin{equation}
\frac{1}{\bar{p}_r^{-1}}\pdv{\bar{p}_r^{-1}}{r}=-\frac{1}{\bar{p}_r}\pdv{\bar{p}_r}{r},
\end{equation}
and the advection term becomes
\begin{equation}
\frac{\bar\varGamma_{0}}{a} =
\int_{0}^{\infty}\qty[\bar{p}_r^{-1}\ev{\mathsf{A'}}_{\xi} -\ev{\mathsf{D'}}_{\xi}
\qty(1-p'\,\bar{p}_r^{-1})\pdv{\bar{p}_r^{-1}}{r}]
\ee^{-p'/\bar{p}_r}\dd{p'}.
\end{equation}
Note that while $\bar{p}_{r}$ has been moved into the integrals, it
does not depend on $p'$.






\subsection{Notes on the normalization}

In \href{https://arxiv.org/abs/2010.07156}{Pontus' paper}, elelctric fields are normalized to the critical
electric field
\begin{equation}
E_{\rm c} = \frac{n_{\ee}e^3\ln\varLambda}{4\pi\epsilon_0^2m_{\ee}c^2}
= \frac{m_{\ee}c}{e\tau}.
\end{equation}




%%%%%%%%%%%%%%%%%%%%%%%%%% The bibliography %%%%%%%%%%%%%%%%%%%%%%%%%%
%\newpage
%% This bibliography uses BibTeX
%\bibliographystyle{ieeetr}
%\bibliography{references}%requires a file named 'references.bib'
%% Citations are as usual: \cite{example_article}




%%%%%%%%%%%%%%%%%%%%%%%%%%%%%%%%%%%%%%%%%%%%%%%%%%%%%%%%%%%%%%%%%%%%%%
\end{document}%% ^ ^ ^ ^ ^ ^ ^ ^ ^ ^ ^ ^ ^ ^ ^ ^ ^ ^ ^ ^ ^ ^ ^ ^ ^ ^ ^
%%%%%%%%%%%%%%%%%%%%%%%%%%%%%%%%%%%%%%%%%%%%%%%%%%%%%%%%%%%%%%%%%%%%%%

%%% Local Variables:
%%% mode: latex
%%% TeX-master: t
%%% End:
